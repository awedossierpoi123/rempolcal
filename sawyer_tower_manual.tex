%generated by fwrite()
\documentclass[a4j,dvipdfmx]{jsarticle}
\usepackage{graphicx}
\usepackage{amsmath, amssymb}
\usepackage{comment}
\usepackage{circuitikz}
\usepackage{here}
\begin{document}
\section{Sawyer-Tower回路}
Sawyer-Tower法は強誘電体の残留分極値を計測するときに使用される測定方法である。
測定の際に使用される回路図を図\ref{fig:sawyertowerimage}に示す。
\begin{figure}[H]
    \centering
    %\ctikzset{bipoles/length=1cm}
    \begin{circuitikz}[scale=0.8,transform shape]
        \draw[] (0,0) to[sV,l=$V$] ++(0,6) to[short] ++(2,0) coordinate(IN1);
        \draw (IN1) -- ($(IN1)+(4,0)$) node[bnc, anchor=zero] (bncin1) {};
        
        \draw[] (IN1) to[C,l=$C_{s}$,name=Cs] ++(0,-3) to[C,l=$C_{0}$,name=C0] ++(0,-3) coordinate(GND);
        \draw[] (0,0) -| (GND);
        \node[] at ($(Cs)+(0.5,0.5)$) {$+Q$};
        \node[] at ($(Cs)+(0.5,-0.5)$) {$-Q$};
        \node[] at ($(C0)+(0.5,0.5)$) {$+Q$};
        \node[] at ($(C0)+(0.5,-0.5)$) {$-Q$};
        \node[] at ($(C0)+(-0.8,0)$) {$V_{0}$};
        \node[] at ($(Cs)+(-0.8,0)$) {$V_{s}$};
        \coordinate(MID) at ($0.5*(Cs)+0.5*(C0)$);
        \draw (MID) -- ($(MID)+(4,0)$) node[bnc, anchor=zero] (bncin2) {};
        \ctikzset{bipoles/oscope/width=1.6}
        \ctikzset{bipoles/oscope/height=1.2}
        \ctikzset{bipoles/oscope/waveform=lissajous}
  
        
        \draw[] ($(bncin1.hot)+(1,1)$) node [oscopeshape,anchor=in 1](OSC) {};
        \node [bnc, anchor=zero,rotate=-90](bnc1) at (OSC.in 1){};
        \node [bnc, anchor=zero,rotate=-90](bnc2) at (OSC.in 2){};
        \draw[] (bncin1.hot) -| (bnc1.hot);
        %\draw[] (MID) -| ($(bnc1.hot)+(0,-1)$) coordinate(END);
        \draw[]  (bncin2.hot) -| (bnc2.hot);
        \node[ground] at (bncin1.shield) {};
        \node[ground] at (bncin2.shield) {};
        \node[ground] at (0,0) {};
        %\draw[] (END) to[short,-o] (OSC2.in 2);
      \end{circuitikz}
    \caption{Sawyer-Tower回路の概略図}\label{fig:sawyertowerimage}
\end{figure}
図\ref{fig:sawyertowerimage}から、
\begin{align}
    Q=C_{s}V_{s}=C_{0}V_{0}
\end{align}
ここで$C_{0}>>C_{s}$とすると、$V_{s}>>V_{0}$となることから、
$V\approx V_{s}$となる。よって、キャパシタ$C_{s}$の強誘電体層の膜厚を$d$とすると、
$E_{s}\approx V/d$となる。
次に、$Q$は電気変位$D$に電極面積$S$をかけたものであることから、
\begin{align}
    V_{0}&=\frac{Q}{C_{0}}=\frac{DS}{C_{0}}\\
    D&=\frac{C_{0}V_{0}}{S}
\end{align}
となる。ここで
\begin{align}
    P&=D-\varepsilon_{0}E_{s}
\end{align}
であるので、
\begin{align}
    P&=\frac{C_{0}V_{0}}{S}-\varepsilon_{0}E_{s}\\
    &\approx \frac{C_{0}V_{0}}{S}-\varepsilon_{0}\frac{V}{d}
\end{align}
である。よって、$V$と$V_{0}$を測るだけで、$P-E$曲線を計算できる。
ただし、rempolcal.pyによる計算では、$E_{s}$を近似せずに計算している。
よって、以下の式で計算を行っている。
\begin{align}
    P&=\frac{C_{0}V_{0}}{S}-\varepsilon_{0}E_{s}\\
    &=\frac{C_{0}V_{0}}{S}-\varepsilon_{0}\frac{V-V_{0}}{d}
\end{align}
\end{document}